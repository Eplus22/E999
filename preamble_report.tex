\geometry{a4paper} 
\usepackage[utf8]{inputenc}
\usepackage{textcomp}
\usepackage{graphicx} 
\usepackage{amsmath,amssymb}  
\usepackage{bm}  
\usepackage{memhfixc} 
\usepackage{pdfsync}  
\usepackage{fancyhdr}
\pagestyle{fancy}
% font
\usepackage{fontspec}
\setmonofont{Consolas}
\usepackage[UTF8]{ctex}
\setCJKmainfont[BoldFont={LXGW WenKai Bold}]{SimSun}
\setmainfont{Times New Roman}
\usepackage[english]{babel}

\usepackage{algorithm} % 伪代码
\usepackage{algpseudocode} % 伪代码

%for abstract
\usepackage{etoolbox}
\patchcmd{\abstract}{\large}{}{}{}

% table
\usepackage{booktabs}
\usepackage{float}

% enumerate itemize
\usepackage{enumitem}
\setlist{nosep} % 对所有的 itemize 和 enumerate 环境应用紧凑设置

% subfigure
\usepackage{subcaption}

\usepackage{array} % 用于减小表中的行间距 \setlength\extrarowheight{-3pt}

\usepackage{xcolor}
% \usepackage{minted}
% \newminted{python}{
%     frame=lines,            % 上下分割线
%     framesep=2mm,           % 代码框与正文的距离
%     baselinestretch=-1,    % 调整代码行间距
%     bgcolor=lightgray,      % 背景颜色
%     linenos,                % 显示行号
%     numbersep=5pt,          % 行号与代码的距离
%     title=\textbf{代码示例},  % 添加标题
%     fontsize=\small,        % 代码字体大小
% }

% For Code
\usepackage{listings}
% 定义通用样式
\lstdefinestyle{commonstyle}{
  keywordstyle=\bfseries,              % 关键字加粗
  basicstyle=\ttfamily,                % 基本字体
  morecomment=[s]{"""}{"""},        % 三重引号作为注释
  commentstyle=\color{gray},           % 注释颜色
  stringstyle=\color{black},           % 字符串颜色正常
  frame=none,                          % 无边框
  showstringspaces=false               % 隐藏字符串中的空格标识
}

% 针对不同语言定义样式
\lstdefinestyle{py}{
  style=commonstyle,                   % 继承通用样式
  language=Python                      % 设置语言为 Python
}

\newcommand{\includecode}[5][py]{
    \lstinputlisting[
        style=#1,
        caption={#2},
        captionpos=b,
        firstline=#3,
        lastline=#4,
        breaklines=true,
        breakatwhitespace=true,
        lineskip=-0.3ex,
        xleftmargin=2em, % 左边距
        xrightmargin=2em % 右边距
    ]{#5}
}

\AtBeginDocument{%
  \renewcommand\contentsname{目录}
  \renewcommand\figurename{图}
  \renewcommand\tablename{目录}
  \renewcommand\lstlistingname{代码}
  \renewcommand{\abstractname}{摘要} % 修改摘要标题
}
