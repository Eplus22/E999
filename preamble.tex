%%% ==================== BASICS ==================== %%%
\usepackage{datetime}
\usepackage[utf8]{inputenc}
\usepackage{textcomp}
\usepackage{url}
\usepackage{graphicx}
\usepackage{float}
\usepackage{booktabs} % \toprule \midrule \bottom rule
\usepackage{enumitem}
\setlist{nosep} % tighten enumerate and itemize
% \usepackage{newunicodechar} % unicode
\usepackage{emptypage}
\usepackage{subcaption}
\usepackage{multicol}
\usepackage{pdfpages}
\usepackage{hyperref}
\hypersetup{
    colorlinks,
    linkcolor={black},
    citecolor={black},
    urlcolor={red!80!black}
}
% indent
\usepackage{indentfirst} % Make sure the first paragraph also indent
% \setlength{\parindent}{2em} % Set paragraph indent 2em

%%% ==================== Math Enhance ==================== %%%
\usepackage{amsmath, amsfonts, mathtools, amsthm, amssymb}
\setcounter{MaxMatrixCols}{30} % FUCKING IDIOT, WHY YOU MOTHERFUCKERS SET THIS RESTRICTION AND THE ERROR MESSAGE IS SOOOOOOOOOOOOOOOOOO UNINFORMATIVE https://tex.stackexchange.com/questions/3519/how-to-use-more-than-10-tab-stops-in-bmatrix-or-other-amsmath-matrix-environment
\usepackage{mathrsfs}
\usepackage{cancel}
\usepackage{bm}
\usepackage{systeme}

%%% ==================== Font & Languages ==================== %%%
\usepackage[T1]{fontenc}
% font
\usepackage{ctex}
\setmainfont{Times New Roman}
% \setCJKmainfont{LXGW WenKai} % All LXGW Wenkai
\setCJKmainfont[BoldFont={LXGW WenKai Bold}]{SimSun}
\setCJKfamilyfont{kai}[AutoFakeBold]{KaiTi}
\newfontfamily\Chancery{URW Chancery L}
\newfontfamily\LXGW{LXGW WenKai}
\usepackage{newtxmath}  % Times New Roman as Math font

%%% ==================== Layout ==================== %%%
% text & bg color (eye-friendly)
% \usepackage[usenames,dvipsnames]{xcolor}
% \definecolor{myBGcolor}{HTML}{F6F0D6}
% % \definecolor{myTextcolor}{HTML}{6B4226} % 4F452C
% \pagecolor{myBGcolor}
% geometry
\usepackage{geometry}
\geometry{
    b5paper,
    left=10mm,
    right=2.5in,
    top=20mm,
    bottom=20mm,
    headheight=15pt,
    headsep=12pt,
    footskip=20pt,    % distance between foot and main content
    % textwidth=120mm, % main text block
    marginparsep=8mm, % gutter between main text block and margin notes
    marginparwidth=45mm % width of margin notes
}
\linespread{1.4}
% fancy headers
\usepackage{fancyhdr}
\pagestyle{fancy}
\fancyhf{}
\fancyhead[RE,LO]{}
\fancyfoot[LE,RO]{\thepage}
\fancyfoot[C]{\leftmark}
\fancyhead[C]{{\Large\Chancery What's done is done.}}


%%% ==================== Custom Commands ==================== %%%
\usepackage{xparse} % extend command defination
% quote environment
\NewDocumentEnvironment{zoe}{}
{
  \par\vspace{\baselineskip}
  \CJKfamily{kai}
  \setlength{\parindent}{4em}
}
{%
  \par\vspace{\baselineskip} 
}
% line spread
\usepackage{setspace}
% marginnote
\usepackage{etoolbox} % condition
\newcommand{\fine}[2][0]{
  \ifdim #1pt=0pt
    \marginnote[1.5\baselineskip]{\setstretch{1.2}\CJKfamily{kai}\small #2}
  \else
    \marginnote[#1\baselineskip]{\setstretch{1.2}\CJKfamily{kai}\small #2}
  \fi
}
% chord
\newcommand\chord[2][l]{%
  \makebox[0pt][#1]{\begin{tabular}[b]{@{}l@{}}#2\\\mbox{}\end{tabular}}}
% poem
\usepackage{verse}
\newcommand{\attrib}[1]{%
\nopagebreak{\raggedleft\footnotesize #1\par}}
\renewcommand{\poemtitlefont}{\normalfont\large\itshape\centering}
% Q.E.D.
\newcommand{\qedz}{\hfill $\blacksquare$}

%%% ==================== Code Environment ==================== %%%
\usepackage{listings}
\lstdefinestyle{commonstyle}{
  keywordstyle=\bfseries,              % bold keyword 
  basicstyle=\ttfamily,                % basic font
  commentstyle=\color{gray},           % comment color
  stringstyle=\color{black},           % stirng color
  frame=none,                          % no frame
  showstringspaces=false               % hiden space indentifier in the string 
}
% 针对不同语言定义样式
\lstdefinestyle{py}{
  style=commonstyle,
  language=python  
}
\lstdefinestyle{cpp}{
  style=commonstyle,
  language=C++
}
\lstdefinestyle{h}{
  style=commonstyle,
  language=C++
}
\lstdefinestyle{c}{
  style=commonstyle,            
  language=C                      
}
\lstdefinestyle{shell}{
  style=commonstyle,            
  language=Shell
}
\usepackage{caption}
\captionsetup[lstlisting]{
    justification=raggedright,  % 左对齐
    singlelinecheck=false
}
\newcommand{\includecode}[5][py]{
    \lstinputlisting[
        style=#1, % default using .py
        caption={#2},
        captionpos=t,
        firstline=#3,
        lastline=#4,
        breaklines=true,
        breakatwhitespace=true,
        lineskip=-0.3ex,
        xleftmargin=2em,
        xrightmargin=2em
    ]{#5}
}

%%% ==================== Titles ==================== %%%
\setcounter{secnumdepth}{3}  % number to subsubsection
\setcounter{tocdepth}{3}     % TOC to subsubsection
\usepackage{titlesec}
\titleformat{\section}
  {\normalfont\Large\bfseries}
  {\thesection}{1em}{}
\titleformat{\subsection}
  {\normalfont\large\bfseries}
  {\thesubsection}{1em}{}
\titleformat{\subsubsection}
  {\normalfont\normalsize\bfseries}
  {\thesubsubsection}{1em}{}
\usepackage{tocloft}
\setlength{\cftbeforesecskip}{6pt}    % section skip
\setlength{\cftbeforesubsecskip}{4pt} % subsection skip

%%% ==================== Theorems Environment ==================== %%%
\makeatother
\usepackage{thmtools}
\usepackage[framemethod=TikZ]{mdframed}
\mdfsetup{skipabove=1em,skipbelow=0em}

\declaretheoremstyle[
    headfont=\bfseries\sffamily\color{ForestGreen!70!black}, bodyfont=\normalfont,
    postheadhook={\leavevmode\\},
    mdframed={
        linewidth=2pt,
        rightline=false, topline=false, bottomline=false,
        linecolor=ForestGreen, backgroundcolor=none, %ForestGreen!5,
    }
]{thmgreenbox}

\declaretheoremstyle[
    headfont=\bfseries\sffamily\color{NavyBlue!70!black}, bodyfont=\normalfont,
    postheadhook={\leavevmode\\},
    mdframed={
        linewidth=2pt,
        rightline=false, topline=false, bottomline=false,
        linecolor=NavyBlue, backgroundcolor=none, %NavyBlue!5,
    }
]{thmbluebox}

\declaretheoremstyle[
    headfont=\bfseries\sffamily\color{RawSienna!70!black}, bodyfont=\normalfont,
    postheadhook={\leavevmode\\},
    mdframed={
        linewidth=2pt,
        rightline=false, topline=false, bottomline=false,
        linecolor=RawSienna, backgroundcolor=none, %RawSienna!5,
    }
]{thmredbox}

\declaretheoremstyle[
    headfont=\bfseries\sffamily\color{purple!80!black}, bodyfont=\normalfont,
    bodyfont=\normalfont,
    postheadhook={\leavevmode\\},
    mdframed={
        linewidth=2pt,
        rightline=false, topline=false, bottomline=false,
        linecolor=purple, backgroundcolor=none,
    }
]{thmpurplebox}

\declaretheoremstyle[
    headfont=\bfseries\sffamily\color{BurntOrange!80!black}, 
    bodyfont=\normalfont,
    postheadhook={\leavevmode\\},
    mdframed={
        linewidth=2pt,
        rightline=false, topline=false, bottomline=false,
        linecolor=BurntOrange, 
        backgroundcolor=none, %BurntOrange!5,
    }
]{thmorangebox}

\declaretheoremstyle[
    headfont=\bfseries\sffamily\color{Red!90!black}, 
    bodyfont=\normalfont,
    postheadhook={\leavevmode\\},
    mdframed={
        linewidth=2pt,
        rightline=false, topline=false, bottomline=false,
        linecolor=Red, 
        backgroundcolor=none, %Red!5,
    }
]{thmalertredbox}

\declaretheoremstyle[
    headfont=\bfseries\sffamily\color{Goldenrod!80!black}, 
    bodyfont=\normalfont,
    postheadhook={\leavevmode\\},
    mdframed={
        linewidth=2pt,
        rightline=false, topline=false, bottomline=false,
        linecolor=Goldenrod, 
        backgroundcolor=none, %Goldenrod!5,
    }
]{thmyellowbox}

\declaretheorem[style=thmgreenbox, numbered=yes, name=Def]{definition}
\declaretheorem[style=thmredbox, numbered=yes, name=Algo]{algo}
\declaretheorem[style=thmbluebox, numbered=yes, name=Eg]{eg}
\declaretheorem[style=thmpurplebox, numbered=yes, name=XEg]{xeg}
\declaretheorem[style=thmredbox, numbered=yes, name=Th]{theorem}
\declaretheorem[style=thmredbox, numbered=yes, name=Lemma]{lemma}
\declaretheorem[style=thmredbox, numbered=yes, name=Cor]{corollary}
\declaretheorem[style=thmblueline, numbered=no, name=Remark]{remark}
\declaretheorem[style=thmalertredbox, numbered=yes, name=Target]{target}
\declaretheorem[style=thmalertredbox, numbered=yes, name=Warning]{warning}
\declaretheorem[style=thmorangebox, numbered=yes, name=Concept]{concept}

\makeatletter
\def\ll@xeg{
  \protect\numberline{\csname the\thmt@envname\endcsname}%
  \ifx\@empty\thmt@shortoptarg
    \thmt@thmname
  \else
    \thmt@shortoptarg
  \fi}
\makeatother

\makeatletter
\def\ll@eg{
  \protect\numberline{\csname the\thmt@envname\endcsname}%
  \ifx\@empty\thmt@shortoptarg
    \thmt@thmname
  \else
    \thmt@shortoptarg
  \fi}
\makeatother

\makeatletter
\def\ll@theorem{
  \protect\numberline{\csname the\thmt@envname\endcsname}%
  \ifx\@empty\thmt@shortoptarg
    \thmt@thmname
  \else
    \thmt@shortoptarg
  \fi}
\makeatother

\makeatletter
\def\ll@definition{
  \protect\numberline{\csname the\thmt@envname\endcsname}%
  \ifx\@empty\thmt@shortoptarg
    \thmt@thmname
  \else
    \thmt@shortoptarg
  \fi}
\makeatother

\makeatletter
\def\ll@algo{
  \protect\numberline{\csname the\thmt@envname\endcsname}%
  \ifx\@empty\thmt@shortoptarg
    \thmt@thmname
  \else
    \thmt@shortoptarg
  \fi}
\makeatother

\makeatletter
\def\ll@concept{
  \protect\numberline{\csname the\thmt@envname\endcsname}%
  \ifx\@empty\thmt@shortoptarg
    \thmt@thmname
  \else
    \thmt@shortoptarg
  \fi}
\makeatother

\makeatletter
\def\ll@target{
  \protect\numberline{\csname the\thmt@envname\endcsname}%
  \ifx\@empty\thmt@shortoptarg
    \thmt@thmname
  \else
    \thmt@shortoptarg
  \fi}
\makeatother

\makeatletter
\def\ll@warning{
  \protect\numberline{\csname the\thmt@envname\endcsname}%
  \ifx\@empty\thmt@shortoptarg
    \thmt@thmname
  \else
    \thmt@shortoptarg
  \fi}
\makeatother

\newcommand{\listthem}[2]{
    \renewcommand{\listtheoremname}{#1}
    \listoftheorems[ignoreall, show={#2}]
}

\AtBeginDocument{
  \renewcommand\contentsname{目录}
  \renewcommand\figurename{Figure}
  \renewcommand\tablename{目录}
  \renewcommand\lstlistingname{CodePieces\ }
  \renewcommand{\abstractname}{摘要} % 修改摘要标题
}
