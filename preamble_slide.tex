\usepackage{datetime}
\usepackage[utf8]{inputenc}
\usepackage[T1]{fontenc}
\usepackage{textcomp}
\usepackage{url}
\usepackage{graphicx}
\usepackage{float}
\usepackage{booktabs} % \toprule \midrule \bottom rule
\usepackage{enumitem}
\setlist{nosep} % tighten enumerate and itemize

% logo
% \usepackage{background}
% \backgroundsetup{
%   scale=1,               % Scale the image
%   color=black,             % Image color
%   opacity=0.1,             % Opacity of the watermark image
%   angle=0,  % rotation
%   position=current page.south east, % This positions the image at the bottom-right corner
%   vshift=1.8cm,            % Adjust vertical shift (distance from the page's bottom edge)
%   hshift=-1.6cm,             % Adjust horizontal shift (distance from the right edge)
%   contents={\includegraphics[width=2cm,height=2cm]{../zx.png}}  % Your image file
% }

\usepackage{newunicodechar} % unicode

\usepackage{emptypage}
\usepackage{subcaption}
\usepackage{multicol}
\usepackage{pdfpages}
\usepackage{hyperref}
\hypersetup{
    colorlinks,
    linkcolor={black},
    citecolor={black},
    urlcolor={red!80!black}
}

\setcounter{secnumdepth}{3}
\setcounter{tocdepth}{3}

\usepackage{amsmath, amsfonts, mathtools, amsthm, amssymb}
\usepackage{mathrsfs}
\usepackage{cancel}
\usepackage{bm}
\usepackage{systeme}

% text & bg color
% \usepackage[usenames,dvipsnames]{xcolor}
% \definecolor{myBGcolor}{HTML}{F6F0D6}
% % \definecolor{myTextcolor}{HTML}{6B4226} % 4F452C
% \pagecolor{myBGcolor}

% font
\usepackage{ctex}
\setmainfont{Times New Roman}
% \setCJKmainfont{LXGW WenKai}
\setCJKmainfont[BoldFont={LXGW WenKai Bold}]{SimSun}
\setCJKfamilyfont{kai}[AutoFakeBold]{KaiTi}
\newfontfamily\Chancery{URW Chancery L} % 定义 Chancery 字体命令
\newfontfamily\LXGW{LXGW WenKai} % 定义 LXGW 字体命令
\usepackage{newtxmath}  % 设置 Times New Roman 作为数学字体
\usepackage{xparse}

\usepackage{fontspec}
\setmonofont{Consolas}

\NewDocumentEnvironment{zoe}{}
{
  \par\vspace{\baselineskip}
  \CJKfamily{kai}
  \setlength{\parindent}{2em}
}
{%
  \par\vspace{\baselineskip} 
}

% line spread
\usepackage{setspace}
% marginnote
\usepackage{etoolbox} % condition
\newcommand{\fine}[2][0]{
  \ifdim #1pt=0pt
    \marginnote[1.5\baselineskip]{\setstretch{1.2}\CJKfamily{kai}\normalsize #2}
  \else
    \marginnote[#1\baselineskip]{\setstretch{1.2}\CJKfamily{kai}\normalsize #2}
  \fi
}

% chord
\newcommand\chord[2][l]{%
  \makebox[0pt][#1]{\begin{tabular}[b]{@{}l@{}}#2\\\mbox{}\end{tabular}}}

% poem
\usepackage{verse}
\newcommand{\attrib}[1]{%
\nopagebreak{\raggedleft\footnotesize #1\par}}
\renewcommand{\poemtitlefont}{\normalfont\large\itshape\centering}

% code
\usepackage{listings}
\lstset{
    columns=fixed,
    basewidth=0.5em,
    basicstyle=\ttfamily,        % 或 \fontfamily{cmtt}\selectfont
    keywordstyle=\bfseries,
    commentstyle=\color{gray},
    stringstyle=\color{black},
    frame=none,
    showstringspaces=false,
    breaklines=true,             % 若需自动换行
    breakatwhitespace=false,     % 允许任意位置断行
    keepspaces=true,             % 保留空格对齐
    % literate={\ }{{\ }}1         % 保留空格字符(防被转义)
    % lineskip=-0.1ex,
    xleftmargin=2em, % 左边距
    xrightmargin=2em % 右边距
}

% 定义通用样式
\lstdefinestyle{commonstyle}{
  keywordstyle=\bfseries,              % 关键字加粗
  basicstyle=\ttfamily,                % 基本字体
  commentstyle=\color{gray},           % 注释颜色
  stringstyle=\color{black},           % 字符串颜色正常
  frame=none,                          % 无边框
  showstringspaces=false               % 隐藏字符串中的空格标识
}
% 针对不同语言定义样式
\lstdefinestyle{py}{
  style=commonstyle,                   % 继承通用样式
  language=Python                      % 设置语言为 Python
}
\lstdefinestyle{cpp}{
  style=commonstyle,                   % 继承通用样式
  language=CPP                      % 设置语言为 C++
}

\newcommand{\includecode}[5][py]{
    \lstinputlisting[
        style=#1,
        caption={#2},
        captionpos=b,
        firstline=#3,
        lastline=#4,
        breaklines=true,
        breakatwhitespace=true,
        lineskip=-0.3ex,
        xleftmargin=2em, % 左边距
        xrightmargin=2em % 右边距
    ]{#5}
}

% geometry
\usepackage{geometry}
\geometry{
    paperwidth=279.4mm, % 16:9 的宽度(11英寸)
    paperheight=157.5mm, % 16:9 的高度(6.2英寸)
    left=15mm,
    right=80mm,
    top=15mm,
    bottom=15mm,
    % textwidth=200mm, % 主内容宽度
    marginparsep=8mm,
    marginparwidth=60mm % 注释边宽度
}

\linespread{1.4}

\usepackage{tocloft}

% 目录的字体大小
\renewcommand{\cftsecfont}{\normalsize}  % 章节标题的字体大小
\renewcommand{\cftsubsecfont}{\normalsize}  % 子章节标题的字体大小
% 目录的行间距
\setlength{\cftbeforesecskip}{0pt}  % 章节标题之前的垂直间距
\setlength{\cftbeforesubsecskip}{0pt}  % 子章节标题之前的垂直间距

\usepackage{titlesec}
\titleformat{\part}
  {\bfseries\Large} % 加粗并设置字体大小
  {\thepart}{0.5em}{} % 章节编号和标题之间的间距
\titleformat{\section}
  {\bfseries\Large} % 加粗并设置字体大小
  {\thesection}{0.5em}{} % 章节编号和标题之间的间距
% 设置 \subsection 标题加粗
\titleformat{\subsection}
  {\bfseries\large}
  {\thesubsection}{0.5em}{} % 子章节编号和标题之间的间距
% 设置 \subsubsection 标题加粗
\titleformat{\subsubsection}
  {\bfseries\normalsize}
  {\thesubsubsection}{0.5em}{} % 子子章节编号和标题之间的间距

% theorems
\makeatother
\usepackage{thmtools}
\usepackage[framemethod=TikZ]{mdframed}
\mdfsetup{skipabove=1em,skipbelow=0em}

\declaretheoremstyle[
    headfont=\bfseries\sffamily\color{ForestGreen!70!black}, bodyfont=\normalfont,
    mdframed={
        linewidth=2pt,
        rightline=false, topline=false, bottomline=false,
        linecolor=ForestGreen, backgroundcolor=none, %ForestGreen!5,
    }
]{thmgreenbox}

\declaretheoremstyle[
    headfont=\bfseries\sffamily\color{NavyBlue!70!black}, bodyfont=\normalfont,
    mdframed={
        linewidth=2pt,
        rightline=false, topline=false, bottomline=false,
        linecolor=NavyBlue, backgroundcolor=none, %NavyBlue!5,
    }
]{thmbluebox}

\declaretheoremstyle[
    headfont=\bfseries\sffamily\color{RawSienna!70!black}, bodyfont=\normalfont,
    mdframed={
        linewidth=2pt,
        rightline=false, topline=false, bottomline=false,
        linecolor=RawSienna, backgroundcolor=none, %RawSienna!5,
    }
]{thmredbox}

\declaretheoremstyle[
    headfont=\bfseries\sffamily\color{purple!80!black}, bodyfont=\normalfont,
    mdframed={
        linewidth=2pt,
        rightline=false, topline=false, bottomline=false,
        linecolor=purple, backgroundcolor=none, % 你可以调整深紫色的明暗
    }
]{thmpurplebox}

\declaretheoremstyle[
    headfont=\bfseries\sffamily\color{RawSienna!70!black}, bodyfont=\normalfont,
    numbered=no,
    mdframed={
        linewidth=2pt,
        rightline=false, topline=false, bottomline=false,
        linecolor=RawSienna, backgroundcolor=none, %RawSienna!1,
    },
    qed=\qedsymbol
]{thmproofbox}

\declaretheoremstyle[
    headfont=\bfseries\sffamily\color{NavyBlue!70!black}, bodyfont=\normalfont,
    numbered=no,
    mdframed={
        linewidth=2pt,
        rightline=false, topline=false, bottomline=false,
        linecolor=NavyBlue, backgroundcolor=none, %NavyBlue!1,
    },
]{thmexplanationbox}

\declaretheorem[style=thmproofbox, name=Proof]{replacementproof}
\renewenvironment{proof}[1][\proofname]{\vspace{-10pt}\begin{replacementproof}}{\end{replacementproof}}

\declaretheorem[style=thmexplanationbox, name=Sol]{tmpexplanation}
\newenvironment{explanation}[1][]{\vspace{-10pt}\begin{tmpexplanation}}{\end{tmpexplanation}}

\declaretheorem[style=thmgreenbox, numbered=yes, name=Def]{definition}
\declaretheorem[style=thmbluebox, numbered=yes, name=Eg]{eg}
\declaretheorem[style=thmpurplebox, numbered=yes, name=XEg]{xeg}
\declaretheorem[style=thmredbox, numbered=yes, name=Th]{theorem}
\declaretheorem[style=thmredbox, numbered=yes, name=Lemma]{lemma}
\declaretheorem[style=thmredbox, numbered=yes, name=Cor]{corollary}
\declaretheorem[style=thmblueline, numbered=no, name=Remark]{remark}

\makeatletter
\def\ll@xeg{%
  \protect\numberline{\csname the\thmt@envname\endcsname}%
  \ifx\@empty\thmt@shortoptarg
    \thmt@thmname
  \else
    \thmt@shortoptarg
  \fi}
\makeatother

\makeatletter
\def\ll@eg{%
  \protect\numberline{\csname the\thmt@envname\endcsname}%
  \ifx\@empty\thmt@shortoptarg
    \thmt@thmname
  \else
    \thmt@shortoptarg
  \fi}
\makeatother

\makeatletter
\def\ll@theorem{%
  \protect\numberline{\csname the\thmt@envname\endcsname}%
  \ifx\@empty\thmt@shortoptarg
    \thmt@thmname
  \else
    \thmt@shortoptarg
  \fi}
\makeatother

\makeatletter
\def\ll@definition{%
  \protect\numberline{\csname the\thmt@envname\endcsname}%
  \ifx\@empty\thmt@shortoptarg
    \thmt@thmname
  \else
    \thmt@shortoptarg
  \fi}
\makeatother

\newcommand{\listthem}[2]{
    \renewcommand{\listtheoremname}{#1}
    \listoftheorems[ignoreall, show={#2}]
}

\newcommand{\qedz}{\hfill $\blacksquare$}

% fancy headers
\usepackage{fancyhdr}
\pagestyle{fancy}
\fancyhf{}
\fancyhead[RE,LO]{}
\fancyfoot[LE,RO]{\thepage}
\fancyfoot[C]{\leftmark}
% \fancyhead[C]{{\Chancery Made with Pain, Joy, and Attention}}
% \fancyhead[C]{{\Chancery ZN Fan 2210377}}
%\fancyhead[C]{{\LXGW 2210377 范倬宁\ 数值分析作业附件}}

% \AtBeginDocument{%
     % \setlength{\parindent}{2em}
     % \setlength{\leftskip}{1em}  % 整体向右偏移 1em
% }
\AtBeginDocument{%
  \fontsize{14}{16}\selectfont
  \renewcommand\contentsname{目录}
  \renewcommand\figurename{图}
  \renewcommand\tablename{目录}
  \renewcommand\lstlistingname{代码}
  \renewcommand{\abstractname}{摘要} % 修改摘要标题
}
